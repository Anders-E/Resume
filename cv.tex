% Original design by Alessandro Plasmati.
% Modified by Anders Eriksson 27 Jan 2016

\documentclass[a4paper,10pt]{article}

\usepackage{marvosym}
\usepackage{fontspec} 						%for loading fonts
\usepackage{xunicode,xltxtra,url,parskip} 	%other packages for formatting
\RequirePackage{color,graphicx}
\usepackage[usenames,dvipsnames]{xcolor}
\usepackage[big]{layaureo} 					%better formatting of the A4 page
\usepackage{titlesec}						%custom \section

%Setup hyperref package, and colours for links
\usepackage{hyperref}
\definecolor{linkcolour}{rgb}{0,0.2,0.6}
\hypersetup{colorlinks,breaklinks,urlcolor=linkcolour, linkcolor=linkcolour}

%FONTS
\defaultfontfeatures{Mapping=tex-text}
\setmainfont[
SmallCapsFont = CrimsonText-Roman.ttf,
BoldFont = CrimsonText-Semibold.ttf,
ItalicFont = Lato-Italic.ttf
]
{Lato-Regular.ttf}

\newfontfamily\bitter[]{CrimsonText-Roman.ttf}

\titleformat{\section}{\Large\scshape\raggedright}{}{0em}{}[\titlerule]
\titlespacing{\section}{0pt}{3pt}{3pt}
%Tweak a bit the top margin
%\addtolength{\voffset}{-1.3cm}

%--------------------BEGIN DOCUMENT-----------------------------
\begin{document}

\pagestyle{empty} % non-numbered pages

\font\fb=''[cmr10]'' %for use with \LaTeX command

%--------------------TITLE--------------------------------------
\par{
    \centering
	{
	    \Huge \bitter Anders Eriksson \\
	    \normalfont
	    \normalsize Computer Science student and programming enthusiast
	}
	\bigskip\par
}

%--------------------SECTIONS-----------------------------------
\section{Summary}
Computer science student at \emph{KTH Royal Institute of Technology} with a huge interest in computer science, human languages, and language technology.

Programming as well as free and open-source software are passions of mine.

\section{Personal Data}

\begin{tabular}{rl}
    \textbf{Place and Date of Birth:} & G{\"a}vle, Sweden  | December 2, 1991 \\
    \textbf{Address:}   & Studentbacken 21-1205, 115 57, Stockholm, Sweden \\
    \textbf{Phone:}     & +46 70 645 24 53\\
    \textbf{email:}     & \href{mailto:eriksson.c.anders@gmail.com
}{eriksson.c.anders@gmail.com}\\
    \textbf{LinkedIn:}	& \href{http://se.linkedin.com/in/canderseriksson}{/in/canderseriksson}\\
    \textbf{GitHub:}	& \href{http://github.com/Anders-E}{/Anders-E}
\end{tabular}

\section{Technical Skills}
\begin{tabular}{r|l}
\textbf{Programming Languages} & Python, Java, C, Perl 5, Lua \\
\textbf{Other Languages} & HTML, CSS, SQL, \LaTeX \\
\textbf{Version Control} & Git \\
\textbf{Other} & Linux\\
\end{tabular}

\section{Experience}
\begin{tabular}{r|p{11cm}}
 \emph{Current, since} & Contributor at \textbf{Duolingo}\\\textbf{December 2014}&\emph{Free Language Education, Voluntary Work}\\&\footnotesize{Been on the team for the course \emph{Swedish for English speakers} since late 2014. The course has more than two million learners as of October 2016. Tasks include developing the course, editing acceptable translations of English
 and Swedish sentences and maintaining the Swedish message boards. Went to the first \emph{Duolingo Incubator Summit} in Berlin in April/May 2015.}\\ 
\end{tabular}

\section{Selected Work}
\begin{tabular}{r|p{11cm}}
\textbf{January - May 2016} & \textbf{WikiPageStats} \\
    & \href{https://github.com/wikimedia/analytics-wikipagestats}{https://github.com/wikimedia/analytics-wikipagestats} \\
    & \footnotesize{Developed a tool for displaying page view statistics from \emph{Wikimedia} projects using their then newly released Pageview API for \emph{Wikimedia Sverige}. The tool was developed using HTML, CSS, JavaScript, AngularJS, and Highcharts. The project was done as part of the course \emph{Software Engineering} at \emph{KTH Royal Institute of Technology}.}\\
&\\
\textbf{December 2015} & \textbf{PIC32 Tetris} \\
    & \href{https://github.com/EmilGedda/PIC32-Tetris}{https://github.com/EmilGedda/PIC32-Tetris} \\
    & \footnotesize{Developed a Tetris clone for a \emph{Microchip PIC32} microcontroller with a classmate for the course \emph{Computer Organization and Components} at \emph{KTH Royal Institute of Technology}. The game was developed in C using the OLED graphic display of a \emph{chipKIT} Basic I/O Shield.}
\end{tabular}

\section{Education}
\begin{tabular}{r|l}
\emph{Graduating in} & \textbf{KTH Royal Institute of Technology}, Stockholm, Sweden\\	
 \textbf{June} 2019 & Master of Science in \textbf{Computer Science}\\&\\
\textbf{Summer} 2015& \textbf{Mid Sweden University}\\& Introduction to \textbf{Operating Systems}, with Applications in Linux\\&\\
\textbf{June} 2013& \textbf{Uppsala University}, Uppsala, Sweden\\& Courses in French, Russian, and linguistics\\&\\
\textbf{June} 2010& \textbf{Vasaskolan}, Gävle, Sweden \\& High School, Natural Science Programme
\end{tabular}

\section{Languages}
\begin{tabular}{rll}
\large\textbf{Language}&\large\textbf{Proficiency}&\large\textbf{Level}\\
\\
\textbf{Swedish:}&Native Language\\
\textbf{English:}&Fluent&(C2)\\
\textbf{French:}&Intermediate&(B1)\\
\textbf{Portuguese:}&Intermediate&(B1)\\
\textbf{Japanese:}&Elementary&(A2)\\
\textbf{Italian:}&Elementary&(A2)\\
\textbf{German:}&Elementary&(A2) \\
\textbf{Russian:}&Elementary&(A2)
\end{tabular}
\end{document}
